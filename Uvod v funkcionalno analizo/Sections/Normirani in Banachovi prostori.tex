\section{Normirani in Banachovi prostori}
\subsection{Definicije in primeri}

\datum{2024-10-2}

\begin{definicija}
    Naj bo $X$ vektorski prostor nad poljem $\F \in \set{\R, \C}$. Preslikava 
    $\norm{\cdot} \colon X \to \R$ je \emph{norma}\index{norma}, če velja
    \begin{enumerate}[(i)]
        \item $\forall x \in X.~\norm{x} \ge 0$,
        \item $\norm{x} = 0 \Rightarrow x = 0$,
        \item $\forall \lambda \in \F.~\forall x \in X.~\norm{\lambda x} = \abs{\lambda} \norm{x}$,
        \item $\forall x, y \in X.~\norm{x+y} \le \norm{x} + \norm{y}$ (trikotniška neenakost).
    \end{enumerate}
\end{definicija}

\begin{definicija}
    Če $p\colon X \to \R$ zadošča lasnostim (a), (c) in (d) iz zgornje definicije, 
    je $p$ \emph{polnorma}\index{polnorma} na $X$.
\end{definicija}

\begin{definicija}
    Prostor $X$ skupaj z normo $\norm{\cdot}$ je \emph{normiran prostor}\index{normiran prostor}.
\end{definicija}

\begin{lema}
    V normiranem prostoru velja
    \[
        \abs{\norm{x} - \norm{y}} \le \norm{x-y}.
    \]
\end{lema}

\begin{proof}
    Iz
    \[
        \norm{x} = \norm{x-y+y} \le \norm{x-y} + \norm{y}
    \]
    sledi
    \[
        \norm{x} - \norm{y} \le \norm{x-y}.
    \]
    Podobno dobimo $\norm{y} - \norm{x} \le \norm{x-y}$.
\end{proof}

Če pišemo $f(x) = \norm{x}$, sledi, da je $f$ zvezna in Lipshitzeva 
s konstanto $1$.

Naj bo $X$ normiran prostor. Vpeljemo metriko
\begin{align*}
    d &\colon X \to \R, \\
    d(x,y) &= \norm{x-y}.
\end{align*}

Omenimo še dve lastnosti metrike $d$:
\begin{itemize}
    \item $d$ je translacijsko invariantna:
    \[
        d(x+y,y+a) = \norm{(x+a) - (y+a)} = \norm{x-y} = d(x,y),
    \]
    \item $d$ je pozitivno homogena:
    \[
        d(\lambda x, \lambda y) = \norm{\lambda x - \lambda y} = 
        \abs{\lambda} \norm{x-y} = \abs{\lambda}d(x,y).
    \]
\end{itemize}

\begin{definicija}
    Normiran prostor $X$ je \emph{Banachov}\index{Banachov prostor}, če je $(X, d)$ poln.
\end{definicija}

\begin{trditev}
    Seštevanje vektorjev in množenje vektorjev s skalarjem sta zvezni operaciji 
    v normiranem prostoru.
\end{trditev}
Domeni seštevanja in množenja sta zaporedoma $X \times X$ in $\F \times X$. Tu mislimo 
zveznost v smislu produktne metrike/topologije.

\begin{proof}
    Naj bo $\varepsilon > 0$. Če velja $\norm{x-x'} < \varepsilon/2$ in 
    $\norm{y-y'} < \varepsilon/2$, potem
    \[
        \norm{(x'+y') - (x+y)} = \norm{(x'-x) - (y'+y)} \le \norm{x'-x} 
        + \norm{y'-y} < \varepsilon.
    \]
    Naj bodo $\varepsilon > 0$, $x \in X$ in $\lambda \in \F$. Velja
    \begin{align*}
        \norm{\lambda' x' - \lambda x} &= \norm{\lambda'x' -\lambda x' + \lambda x' - \lambda x} \\
        &\le \abs{\lambda'-\lambda} \norm{x'} + \abs{\lambda} \norm{x'-x}.
    \end{align*}
    Naj bo\footnote{V imenovalcu dodamo $+1$ zato, da ne potrebujemo ločeno obravnavati primera $\lambda = 0$.} 
    $\norm{x'-x} \le \frac{\epsilon}{2} \cdot \frac{1}{\abs{\lambda} + 1}$. Tedaj velja
    \[
        \norm{x'} = \norm{x' -x +x} \le \norm{x'-x}+ \norm{x} \le \frac{\epsilon}{2} \cdot \frac{1}{\abs{\lambda} + 1} + \norm{x}.
    \]
    Če velja še
    \[
        \norm{\lambda' - \lambda} < \frac{\epsilon}{2} 
        \left(\frac{\epsilon}{2} \cdot \frac{1}{\abs{\lambda} + 1} + \norm{x}\right)^{-1},
    \]
    dobimo $\norm{(x'+y') - (x+y)} < \epsilon$.
\end{proof}

\datum{2024-10-7}

\begin{primer}
    Poglejmo $(\F^n, \norm{\cdot}_p)$, kjer je $1 \le p \le \infty$.

    Za $1 \le p < \infty$ in $x = (x_1, \dots, x_n)$ definiramo
    \begin{align*}
        \norm{x}_p &= \left(\sum_{k=1}^n \abs{x_k}\right)^{1/p}, \\
        \norm{x}_\infty &= \max_{1 \le k \le n} \abs{x_k}.
    \end{align*}
    % Narisali smo kako izgledajo krogle v n=2. B_inty je kvadrat
    % B_1 krog, B_p gre proti kvadratu, ko gre p proti infty
    Velja, da so $(\F^n, \norm{\cdot}_p)$ Banachovi prostori
    (dokazali smo na vajah).
    
    Vse odprte krogle $B(x, r)$ in zaprte krogle so vedno 
    konveksne (DN, v primeru $0 < p < 1$ to ni res,
    zato $\norm{\cdot}$ ni norma na $F^n$ za $0 < p < 1$).
    %nariši slikice v Desmosu
\end{primer}

\begin{definicija}
    Množica $A$ nad $\F$ je \emph{algebra},\index{algebra} če velja
    \begin{enumerate}[(i)]
        \item $(A, +, \cdot)$ je kolobar.
        \item $A$ je vektorski prostor nad $\F$.
        \item $\forall x,y\in A.~\forall \lambda \in \F.~\lambda(xy) = (\lambda x)y = x(\lambda y)$.
    \end{enumerate}
    Če ima $A$ enoto, je $A$ \emph{unitalna}\emph{unitalna algebra}. Če je 
    $A$ tudi normiran prostor, potem je normirana algebra,\index{normirana algebra}
    če velja \emph{submultiplikativnost}\index{submultiplikativnost}
    \[
        \norm{x y} \le \norm{x} \cdot \norm{y}.
    \]
    Če ima normirana algebra enoto $e$, potem zahtevamo, da 
    je $\norm{e} = 1$.
\end{definicija}

\begin{trditev}
    V normirani algebri je množenje zvezna operacija.
\end{trditev}

\begin{proof}
    Podobno kot pri zveznosti množenja s skalarjem.
\end{proof}

\begin{primer}
    Naj bo $X$ Hausdorffov topološki prostor in
    \[
        C_b(X) =\set{\text{zvezne omejene funkcije iz $X$ v $\F$}}.
    \]
    Operacije definiramo po točkah:
    \begin{align*}
        (f+g)(x) &= f(x) + g(x), \\
        (\lambda f)(x) &= \lambda f(x), \\
        (f\cdot g)(x) &= f(x) g(x).
    \end{align*}
    Sledi, da je $C_b(X)$ algebra. Dokažimo, da 
    je normirana algebra in hkrati Banachov prostor, 
    torej \emph{Banachova algebra}\index{Banachova algebra} 
    glede na normo
    \[
        \norm{f}_\infty = \sup_{x \in X} \abs{f(x)}.
    \]
    Dokažimo, da je to res norma. 
    Ker so funkcije omejene je supremum dobro definiran, 
    torej drži (i). Očitno držita tudi (ii) in (iii).

    Naj bo $x \in X$. Velja
    \begin{align*}
        \abs{f(x) + g(x)} &\le \abs{f(x)} + \abs{(g(x))} \\
        &\le \norm{f}_\infty + \norm{g}_\infty.
    \end{align*}
    Sledi
    \[
        \norm{f + g}_\infty \le \norm{f}_\infty + \norm{g}_\infty.
    \]
    Torej je $\norm{\cdot}_\infty$ norma. Dokažimo, da je prostor Banachov.
    Naj bo $(f_n)_{n \in \N}$ Cauchyjevo zaporedje v $C_b(X)$. Tedaj velja
    \[
        \forall \varepsilon > 0.~\exists N_\varepsilon.~\forall n,m \ge n_\varepsilon.~
        \norm{f_n - f_m}_\infty < \varepsilon.
    \]
    Torej za $n,m \ge n_\varepsilon$ velja
    \[
        \forall x \in X. \abs{f_n(x) - f_m(x)}_\infty < \varepsilon.
    \]
    Dobimo, da je $(f_n(x))_{n\in\N}$ Cauchyjevo v $\F$.
    Naj bo
    \[
        f(x) := \lim_{n\to \infty} f_n(x),
    \]
    torej $f \colon X \to \F$. To je edini kandidat za limito. Preveriti moramo
    \[
        f \in C_b(X) \quad \text{in} \quad f_n \to f \text{ glede na $\norm{\cdot}_\infty$}
    \]
    Velja
    \[
        \forall x \in X. \forall n,m \ge n_\varepsilon. \abs{f_n(x) - f_m(x)} < \varepsilon
    \]
    Sledi (ko gre $n \to \infty)$: 
    \begin{equation}
        \forall x \in X. \forall m \ge n_\varepsilon.~\in X.~\abs{f(x) - f_m(x)} \le \varepsilon-
    \end{equation}
    Torej je $f - f_m$ omejena in zato je tudi $f = (f - f_m) + f_m$ omejena.

    Pokažimo, da je $f$ zvezna.
    Izberimo $m \ge n_\varepsilon$. Dokažimo, da je $f$ zvezna 
    v $x \in X$.
    \begin{align*}
        \abs{f(y)-f(x)} &= \abs{f(y)-f_m(y)+f_m(y)-f_m(x)+f_m(x)-f(x)} \\
        &\le |f(y) - f_m(y)| + \abs{f_m(y) - f_m(x)} + \abs{f_m(x) - f(x)} \\
        &\le \varepsilon + \abs{f_m(y) - f_m(x)} + \varepsilon    
    \end{align*}
    Ker je $f_m$ zvezna, obstaja odprta okolica $U$ za $x$, 
    na kateri 
    \[
        \abs{f_m(y) - f_m(x)} < \varepsilon.
    \]
    Torej je $f$ zvezna. 

    Za $m \ge n_\varepsilon$ v $\star$ % kvadratek na sliki
    naredimo supremum po $x \in X$.
    \[
        \norm{f - f_m}_\infty \le \varepsilon
    \]
    Sledi, da $f_n \to f$ v $C_b(X)$ oziroma $C_b(X)$ 
    je Banachov.

    Dokazati moramo, da je norma submultiplikativna in 
    $\norm{1}_\infty = 1$.
    Velja
    \[
        \forall x \in X.~ \abs{f(x)g(x)} \le \norm{f}_\infty\norm{g}_\infty,
    \]
    torej
    \[
        \norm{fg} \le \norm{f} \cdot \norm{g}.
    \]
    Velja tudi
    \[
        \norm{1}_\infty = \sup_{x \in X} \abs{1(x)} = \sup_{x \in X} 1 = 1.
    \]
\end{primer}

\begin{opomba}
    V resnici Hausdorffovosti v zgornjem primeru ne potrebujemo. Včasih jo 
    vseeno predpostavimo, sploh ko imamo opravka z lokalno kompaktnostjo, da 
    ne pride do nesoglasij glede definicije.
\end{opomba}

\begin{posledica}
    Če je $X$ kompakten Hausdorffov prostor, potem je $C(X)$ 
    Banachova algebra.
\end{posledica}

\begin{proof}
    Velja $C(X) = C_b(C)$.
\end{proof}

\begin{primer}
    Prostor $X = \N$ opremimo z diskretno topologijo. Funkcije so 
    v tem primeru kar zaporedja:
    \begin{align*}
        f \colon \N &\to \F, \\
        f &\leftrightarrow (f(1), f(2), \dots).
    \end{align*}
    Podobno priredimo
    \begin{align*}
        C(\N) \leftrightarrow \F^{\N} \\
        C_b(\N) \leftrightarrow l^\infty
    \end{align*}
    Tu je $l^\infty$ prostor omejenih zaporedij za normo
    \[
        \norm{(x_1, x_2, \dots)} = \sup_{x \in X}\abs{x_n}.
    \] 
\end{primer}

\begin{trditev}
    Naj bo $X$ normiran prostor in $Y$ podprostor v $X$. 
    \begin{enumerate}
        \item Če je $Y$ poln, je $Y$ zaprt v $X$.
        \item Če je $X$ Banachov, potem je $Y$ Banachov
        natanko tedaj, ko je zaprt.
    \end{enumerate}
\end{trditev}

\begin{opomba}
    Podprostore normiranega prostora $X$ vedno opremimo z normo 
    definirano na $X$. Topologija je relativna.
\end{opomba}

\begin{proof}
    (i): Naj bo $y \in \overline{Y}$. Dokažimo $y \in Y$. 
    Obstaja $(y_n)_{n \in \N}$ v $Y$, da $y_n \to y$. Sledi, 
    da je $(y_n)$ Cauchyjevo v $Y$. Ker je prostore poln,
    konvergira proti $y'$ v $Y$. Zaradi enoličnosti limite,\footnote{To sledi iz Hausdorffovosti.}
    sledi $y = y' \in Y$.

    (ii): Naj bo $Y$ zaprt v $X$. Pokažimo, da je Banachov.
    Naj bo $(y_n)$ Cauchyjevo v $Y$. Torej je Cauchyjevo 
    v $X$. Torej konvergira proti $y \in X$, saj je $X$ 
    poln. Ker je $Y$ zaprt, je $y \in Y$, torej 
    $y_n \to y$ in $Y$.
\end{proof}


    Naj bo $X$ lokalno kompakten Hausdorffov prostor. Definiramo
    \[
        C_0(X) := \setb{f \in C(X)}{\forall \varepsilon > 0.~
        \exists K^{\text{komp.}} \subseteq X.~\forall x 
        \in X\setminus K.~\abs{f(x)} < \varepsilon}.
    \]
    Tedaj $f \in C_0(X)$ pomeni vse zvezne funkcije, ki 
    gredo proti $0$ v ``neskončnosti''.

\begin{primer}
    Na realni premici se to prevede v
    \[
        f \in C_0(\R) \Leftrightarrow \lim_{\abs{x} \to \infty} \abs{f(x)}= 0.
    \]
    Pri naravnih številih, $C_0(\N)$, pa dobimo
    \[
        c_0 := \setb{(x_1,x_2, \dots)}{\lim_{n \to \infty} x_n = 0}
    \]
\end{primer}

Dokažimo, da je $C_0(X)$ zaprt dvostranski ideal v $C_b(X)$.
Dokažimo, da je $C_0(X)$ zaprt podprostor. 

(i) Naj bosta $f,g \in C_0(X)$. Potem $f + g \in C_0(X)$ in velja
\[
    \forall \varepsilon > 0. \exists K_1, K_2~\text{kompaktna}.~
    \abs{f(x)} < \varepsilon/2
\]
za $x \in X\setminus K_1$ in 
\[
    \abs{g(x)} < \frac{\varepsilon}{2}
\]
za $x \in X \setminus K_2$. Za $x \in X \setminus (K_1 \cup K_2)$ je
\[
    \abs{(f+g)(x)} \le \abs{f(x)} + \abs{g(x)} < \frac{\varepsilon}{2} + \frac{\varepsilon}{2} = \varepsilon.
\]
Pokazali smo $f + g \in X_0(X)$.

Dokažimo $\lambda f \in C_0(X)$.
\[
    \forall \varepsilon > 0.~\exists K^{komp.}.~
    \abs{f(x)} < \frac{\varepsilon}{\abs{\lambda} + 1}
\]
za vse $x \in X \setminus K$. Torej za $x \in X\setminus K$
\[
    \abs{\lambda f(x)} = \abs{\lambda}
    \abs{f(x)} < \frac{\abs{\lambda}}{\abs{\lambda} + 1} \cdot \varepsilon
    < \varepsilon,
\]
oziroma $\lambda f \in C_0(X)$.

Dokažimo, da je $C_0(X)$ zaprt v $C_b(X)$.
Naj bo $(f_n)_{n\in \N}$ v $C_0(X)$ in $f_n \to f$ v $C_b(X)$.
Dokažimo $f \in C_0(X)$. Imamo
\[
    \forall \varepsilon > 0.~\exists n_\varepsilon.~\forall 
    n \ge n_\varepsilon.~\norm{f-f_n}_\infty < \varepsilon.
\]
V posebnem torej
\[
    \norm{f - f_{n_\varepsilon}} < \varepsilon.
\]
Ker je $f_{n_\varepsilon} \in C_0(X)$, obstaja $K^{\text{komp.}} \subseteq X$, 
da je 
\[
    \forall x \in X \setminus K.~\abs{f_{n_\varepsilon}(x)} < \varepsilon.
\]
Za $x \in X\setminus K$ velja
\[
    \abs{f(x)} \le \abs{f(x) - f_{n_\varepsilon}(x)} + \abs{f_{n_\varepsilon}} 
    < \epsilon + \epsilon = 2 \epsilon.
\]
Torej $f \in C_0(X)$.

Pripomnimp še zakaj je $C_0 \subseteq C_b(X)$. Torej zakaj je 
$f \in C_0(X)$ vedno omejena. Po definiciji je izven nekega
kompakta omejena. Na kompaktu pa je $f$ omejena zaradi zveznosti.

Preostane še, da je $C_0(X)$ dvostranski ideal v $C_b(X)$.
Ker je algebra $C_b(X)$ komutativna, preverimo, da je levi ideal. 
Naj bo $f \in C_b(X)$ in $g \in C_0(X)$. Naj bo $\varepsilon > 0$.

Ker je $g \in C_0(X)$, obstaja kompakt $K$ v $X$, da je 
\[
    \abs{g(x)} < \frac{\varepsilon}{\norm{f}_\infty + 1}
\]
za $x \in X \setminus K$. Torej za $x \in X \setminus K$ velja
\[
    \abs{f(x)g(x)} <  \abs{f(x)} \frac{\varepsilon}{\norm{f}_\infty + 1} 
    \le \frac{\norm{f}_\infty}{\norm{f}_\infty + 1} \varepsilon < \varepsilon
\]
Sledi, da je $C_0(X)$ Banachova algebra, ki nima nujno enice, saj velja
\[
    C_0(X)~\text{ima enoto} \Leftrightarrow X~\text{je kompakt}.
\]
\begin{primer}
    Prostor vseh zaporedij, ki konvergirajo proti $0$, $c_0$, je 
    Banachov prostor oziroma celo Banachova algebra.
\end{primer}

\begin{opomba}
    Prostor $l^p$, kjer $1 \le p < \infty$, definiramo kot
    \[
        l^p := \setb{(x_1, x_2, \dots)}{ \sum_{n=1}^{\infty} \abs{x}^p < \infty}.
    \] 
    Tega prostora ne dobimo iz zveznih funkcij. Dobimo ga iz 
    teorije mere. Velja, da je 
    $(l_p, \norm{\cdot}_\infty)$ je Banachov za 
    \[
        \norm{x}_p = \left(\sum_{n=1}^{\infty}\abs{x_n}^p\right)^{1/p}.
    \]
\end{opomba}

\begin{primer}
    Naj bo
    \[
        c = \setb{(x_1, x_2, \dots)}{\text{limita $x_n$ obstaja}}
    \]
    in $\norm{x}_\infty = \sup_{n \in \N} \abs{x_n}$. Sledi, da je
    $(c, \norm{\cdot}_\infty)$ Banachov.

    Definiramo še 
    \[
        c_{00} = \setb{(x_1, x_2, \dots)}{x_n = 0~\text{od nekod dalje}}.
    \]
    Velja $c_{00} \subseteq c_0 \subseteq c \subseteq l^\infty$ in 
    $c_{00} \subseteq l^{p}$ za $1 \le p < \infty$. Več o teh 
    prostorih smo povedali na vajah.
\end{primer}

\subsection{Napolnitve normiranih prostorov}

Naj bo $X$ normiran prostor. Ali obstaja Banachov prostor 
$\hat X$, da je $X \subseteq \hat X$. Izkaže se, da ja,  
$\hat X$ lahko celo izberemo tako, da bo $X$ gost v $\hat X$.

\datum{2024-10-09}