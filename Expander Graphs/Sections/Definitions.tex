\section{Definitions}

\subsection{Expansion}

\datum{2024-10-1}

\begin{definition}
    Let $P = (V ,E)$ be a finite graph.
    \begin{enumerate}[(1)]
        \item For any disjoint subsets of vertives $V_1,V_2 \subseteq V$, 
        we denote by $\EE(V_1, V_2)$ the set of edges of $\Gamma$ with 
        one extremity in $V_1$ and the other in $V_2$. 
        We denote $\EE(V_1) := \EE(V_1, V \setminus V_1)$.
        \item The \emph{Cheeger constant}\index{Cheeger constant} or \emph{expansion constant}\index{expansion constant} 
        of $\Gamma$ is
        \[
            h(\Gamma) := \min\setb{\frac{\abs{\EE(W)}}{\abs{W}}}
            {\emptyset \neq W \subseteq V \land \abs{W} \le \frac{1}{2}\abs{V}}
        \]
        with the convention that $h(\Gamma) = +\infty$ if $\Gamma$ has 
        at most one vertex.
    \end{enumerate}
\end{definition}

The larger $h(P)$ is, the more difficult it is to disconnect a large subset of $V$ from 
the rest of the graph. This will be our way of measuring ``high connectivity''.

\begin{lemma}
    Let $\Gamma$ be a finite graph with at least two vertices (so that $h(\Gamma) < \infty$).
    We have $h(\Gamma) > 0$ if and only if $\Gamma$ is connected.
\end{lemma}

\begin{proof}
    The condition $h(\Gamma) = 0$ means that there exists some nonempty $W \subseteq V$, 
    $\abs{W} \le \frac{1}{2}\abs{V}$, such that $\abs{\EE}(W) = 0$. Since $W \neq V$,
    there is no path between $W$ and $V \setminus W$, so $\Gamma$ is disconnected.
    
    Conversely, if $\Gamma$ is disjoint, there is some componenet, say $W$, of size 
    at most $\frac{1}{2} \abs{V}$. Hence, $\EE(W) = \emptyset$ and $h(\Gamma) = 0$.
\end{proof}

\datum{2024-10-07}

\begin{example}
    Let $K_m$ be the complete graph on $m$ vertieces.
    We have
    \[
        h(K_m) = \min_{1 \le j \le \frac{m}{2}} 
        \frac{\abs{\EE(\set{1,2, \dots, j})}}{j} = 
        \min_{1 \le j \le \frac{m}{2}} 
        \frac{j(m-j)}{j} = m - \floor{\frac{m}{2}}.
    \]
    This goes do infinity, as $m$ goes to infinity. This 
    unsurprisingly tells us that complete graphs are very
    well connected.
\end{example}

\begin{example}
    Let $C_m$ be the $m$-cycle. We have
    \[
        h(\Gamma) = \frac{2}{\floor{\frac{m}{2}}}.
    \]
    This goes to zero.
\end{example}

\begin{example}
    Let $T_{d, k}$ be the $d$-regular tree of depth 
    $k$, where $d \ge 2$. Let $W$ be one 
    of the subtrees hanging from the root. We have
    \[
        h(T_{d,k}) \le \frac{\abs{\EE(W)}}{\abs{W}} 
        = \frac{1}{\frac{\abs{T}-1}{d}} = 
        \frac{d}{\abs{T}-1}.
    \]
    This is a small upper bound.
\end{example}

\subsubsection*{Bounding}

\begin{remark}
    It is much easier to give upper bounds for $h$
    than lower bounds.
\end{remark}

Let's calculate some trivial bounds. Let $\Gamma$ be
connected. Then
\[
    \frac{2}{\abs{\Gamma}} \le h(\Gamma) 
    \le \min_{x\in V}\val(x)
\]
For the first inequality we use 
$\frac{\abs{\EE(W)}}{\abs{W}} \ge
 \frac{1}{\abs{V}/2}$,
and for the second, we take $W = \set{x}$.

Our goal will be to find graphs with large 
$h(\Gamma)$ and few edges (unlike $K_m$).

\subsubsection*{Conncetivity with diameter}

\begin{proposition}
    Let $\Gamma$ be connected. 
    Let $v = \max_{x \in V} \val(x)$. Then
    \[
        \diam\Gamma \le 2 \cdot 
        \frac{\log\frac{\abs{\Gamma}}{2}}
        {\log(1+\frac{h(\Gamma)}{V})} + 3.
    \]
\end{proposition}

\begin{proof}
    Let $x \in V$. For $n \in \N$, consider the ball 
    \[
        B_x(v) =  \setb{y \in V}{d(x,y) \le n}.
    \]
    Claim: 
    \[
        \abs{B_x(n)} \ge \min\set{\frac{\abs{\Gamma}}{2}, \left(1+\frac{h(\Gamma)}{v}\right)^n}
    \]% TO DO CLAIM sty
    \begin{proof}
        We use induction on $n$. Suppose 
        $\abs{B_x(n)} < \frac{\abs{\Gamma}}{2}$ and take 
        $W = B_x(n)$. Note 
        \begin{align*}
            \abs{B_x(n+1) \setminus B_x(n)} 
            &\ge \frac{\abs{\EE(B_x(n))}}{v} \\
            &\ge \frac{h(\Gamma) \abs{B_x(n)}}{v}.
        \end{align*}
        Hence,
        \[
            \abs{B_x(n+1)} \ge
            \left(1 + \frac{h(\Gamma)}{v}\right)\abs{B_x(n)}. \qedhere
        \]        
    \end{proof}
    Let's use the claim to prove the proposition. 
    Let $x,y \in V$. Let $n$ be smallest such that
    \[
        \left(1 + \frac{h(\Gamma)}{v}\right)^n \ge
        \frac{\abs{\Gamma}}{2}. % write n = with log (look galery)
    \]
    This means 
    \[
    \abs{B_x(n)} \ge \frac{\abs{\Gamma}}{2}, \qquad
    \abs{B_y(n)} \ge  \frac{\abs{\Gamma}}{2}.
    \]
    Hence,
    \[
        \abs{B_x(n+1) \cap B_y(n)} > 0.
    \]
    Hence, $d(x,y) \le 2n + 1$.
\end{proof}

\subsubsection*{Expander graphs}

\begin{definition}
    A family $(\Gamma_i)_{i \in \N}$ of 
    graphs is an \emph{expander family}\index{expander family}
    if there are constants $v \ge 1$, $h > 0$ such that
    \begin{enumerate}
        \item $\abs{V_i} \to \infty$ % i to infty over line
        \item $\forall i.~\max_{x \in V}\val(x) \le v$
        \item $\forall i.~h(\Gamma_i) \ge h$.
    \end{enumerate}
\end{definition}

% graphs grow
% absolutely bounded valency ... sparse
% absolutely vounded cheegere constant ... gihly connected

\begin{remark}
    The second condition implies $\abs{E_i} \le v \abs{V_i}$, 
    so there are only linearly many edges in $\Gamma_i$. 
    This gives us a low cost of construction (in applications).
\end{remark}

\subsubsection*{Diameters of expanders}

Let $(\Gamma_i)_{i \in \N}$ be an expander family. Then
\[
    \diam \Gamma_i \ll \log\abs{V_i},
\]
where the implicit constant depends only on $(v, h)$, 
so it is absolute in the family.

%Notation: $f, g \colon \N \to \N$. 
%\[
%    f \ll g \Leftrightarrow g = O(g) \Leftrightarrow 
%    f(n) \le Cg(n) 
%\]
%for all large enough $n$.

This fact follows immediatly from the previous proposition.

\begin{remark}
    Having $\log$-diameter is not the same as being
    an expander family.
\end{remark}

\begin{example}
    The tree $T_{d,k}$ has $\log$-diameter and it 
    does have bounded valency (for fixed $d$), but 
    the Cheeger constant goes to $0$ as $k$ goes 
    to infinity.
\end{example}

\datum{2024-10-08}