\documentclass[12pt, a4paper]{article}

\usepackage[en,black]{../../general}

\begin{document}
\Huge
\begin{center}
    Expander Graphs \\
    \Large
    List of exercises
\end{center}
\normalsize
All the exercises are taken from An Introduction to Expander Graphs by E.~Kowalski % cite properly
(the reference to the book is in brackets).

\setcounter{section}{1}
\section{Graphs}

\textbf{Exercise 2.1 (Exercise 2.1.3)}
Show that if $\Gamma = (V, E, \epsilon)$ is a finite graph, then
\[
\sum_{x \in V} \text{val}(x) = 2|E_2| + |E_1|
\]
where $E_i = \{\alpha \in E \mid \alpha \text{ has } i \text{ extremities}\}$, i.e., $|E_1|$ is the number of loops and $|E_2|$ is the number of edges joining distinct vertices.

\textbf{Exercise 2.2 (Exercise 2.1.7)}
Write the adjacency matrices of the following graphs:
\begin{itemize}
    \item[(i)] Cycles with 4 and 5 vertices.
    \item[(ii)] Paths with 4 and 5 vertices.
    \item[(iii)] Complete graphs with 3, 4, and 5 vertices.
    \item[(iv)] Cayley graph of the symmetric group $S_3$ with the generating sets $\{(12), (13)\}$ and $\{(12), (13), (23)\}$.
\end{itemize}

\textbf{Exercise 2.3 (Exercise 2.1.19)}
\begin{itemize}
    \item[(i)] Let $\Gamma$ be a finite $d$-regular graph with girth $g \geq 3$. Prove that
    \[
    |\Gamma| \geq d(d-1)^{\lfloor \frac{g-3}{2} \rfloor}.
    \]
    \item[(ii)] Show that the girth of a finite $d$-regular graph $\Gamma$ with $d \geq 3$ is $\ll \log(|\Gamma|)$, where the implied constant depends only on $d$.
\end{itemize}

\textbf{Exercise 2.4 (Exercise 2.1.19)}
Show that the number of vertices and edges of the finite tree $T_{d,k}$ are given by
\[
|T_{d,k}| = d \frac{(d-1)^k - 1}{d-2} + 1, \quad |E_{d,k}| = |T_{d,k}| - 1 = d \frac{(d-1)^k - 1}{d-2}
\]
if $d \geq 3$, and $|T_{2,k}| = 2k + 1$, $|E_{2,k}| = 2k$.

\textbf{Exercise 2.5 (Forests and Trees (Exercise 2.2.7))}
Recall that forests are graphs of infinite girth, and trees are connected forests.
\begin{itemize}
    \item[(i)] Show that the diameter of a finite tree $T_{d,k}$ with $d \geq 2$ and $k \geq 0$ is equal to $2k$, and is achieved by the distance between any two distinct vertices labeled with words $(s_1, \dots, s_k)$ and $(s'_1, \dots, s'_k)$ of (maximal) length with $s_1 \neq s'_1$.
    \item[(ii)] Show that if $T$ is a tree, then for any two vertices $x$ and $y$, there exists a unique geodesic on $T$ with endpoints $x$ and $y$ (the image of all paths of length $d_T(x, y)$ between two vertices $x$ and $y$ in $T$ is the same).
    \item[(iii)] If $T = T_{d,k}$ with the ``root'' vertex $x_0 = \emptyset$ and $0 \leq j \leq k$, show that
    \[
    V' = \{x \in V_T \mid d_T(x_0, x) \leq j\}
    \]
    induces a full subgraph isomorphic to $T_{d,j}$.
    \item[(iv)] If $T = T_{d,k}$ with root $x_0$ and $x \in T$ is any vertex, show that
    \[
    V'' = \{y \in V_T \mid d_T(y, x_0) \geq d_T(y, x)\}
    \]
    induces a full subgraph $T''$ of $T$ which is also a tree.
    \item[(v)] Let $\Gamma$ be any graph with girth $\ell \geq 1$, and let $x_0 \in V$. Show that the subgraph of $\Gamma$ induced by
    \[
    V' = \{x \in V \mid d_\Gamma(x_0, x) < \frac{\ell}{2}\}
    \]
    is a tree.
\end{itemize}

\textbf{Exercise 2.6 (Exercise 2.2.9)}
Let $k \geq 2$ be an integer, and let $G = S_k$ be the symmetric group on $k$ elements. Suppose we are given a subgroup $H$ of $G$ such that:
\begin{itemize}
    \item $H$ acts transitively on $\{1, \dots, k\}$;
    \item $H$ contains at least one transposition;
    \item $H$ contains a cycle $\sigma$ of length $p > k/2$ such that $p$ is prime.
\end{itemize}
Prove that $H = G = S_k$.

Let $\Gamma = (V, E)$ be the simple graph with $V = \{1, \dots, k\}$ and an edge between any pair $(i,j) \in V \times V$ such that $i \neq j$ and the transposition $(ij)$ is in $H$. The second assumption ensures that the edge set is not empty.
\begin{itemize}
    \item[(i)] Show that any connected component in $\Gamma$ is a complete graph.
    \item[(ii)] Show that it is enough to prove that $\Gamma$ is connected to conclude that $H = G$.
    \item[(iii)] Show that the action of $G$ on $\{1, \dots, k\}$ induces an action of $G$ on $\Gamma$ by automorphisms. Then show that $G$ acts transitively on the set of all connected components of $\Gamma$. Deduce that all such components are isomorphic.
    \item[(iv)] Show that the $p$-cycle $\sigma \in H$ must fix (globally, not necessarily pointwise) each component of $\Gamma$, and conclude from this.
\end{itemize}



\end{document}