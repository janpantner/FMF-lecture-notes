\section{Independence, matching, covers}

\subsection{Definitions}

\datum{2024-10-02}

Let $G = (V, E)$ be a graph.

\begin{definition}
    A set $S \subseteq V$ is an \emph{independent set}\index{independent set} if $G[S]$ contains no edges.
    The \emph{independence number $\alpha(G)$}\index{independence number} is the maximum possible cardinality of an independent set.
\end{definition}

\begin{definition}
    A set $T \subseteq V$ is a \emph{vertex cover}\index{vertex cover} if 
    \[
        \forall e \in E.~T\cap e \neq \emptyset.
    \]
    The \emph{vertex cover number} $\beta(G)$\index{vertex cover number} is the minimum 
    possible cardinality of a vertex cover.
\end{definition}

\begin{definition}
    A set $M \subseteq E$ is a \emph{matching}\index{matching} if
    \[
        \forall e,f \in M.~e \neq f \Rightarrow e \cap f = \emptyset.
    \]
    The \emph{matching number} $\alpha'(G)$\index{matching number} is 
    the maximum possible cardinality of a matching.
\end{definition}

\begin{definition}
    A set $C \subseteq E$ is an edge cover if every vertex of $G$ is covered 
    by at least one edge from $C$. If $\delta(G) \ge 1$, we define the 
    \emph{edge cover number} $\beta'(G)$\index{edge cover number} as 
    the minimum possible cardinality of an edge cover.
\end{definition}

\begin{lemma}
    Let $G$ be a graph. The following holds:
    \begin{itemize}
        \item $\alpha(G) + \beta(G) = n(G)$.
        \item $\alpha'(G) \le \beta(G)$.
        \item $\alpha(G) \le \beta'(G)$.
        \item If $\delta(G) \ge 1$, then $\alpha'(G) \le \frac{n}{2} \le \beta(G)$.
    \end{itemize}
\end{lemma}

%\begin{proof}
%    TO DO
%\end{proof}

\begin{theorem}[Gallai]\index{Gallai's theorem}
    If $\delta(G) \ge 1$, then $\alpha'(G) + \beta'(G) = n(G)$.
\end{theorem}

%\begin{proof}
%    TO DO
%\end{proof}

\begin{definition}
    Let $M$ be a matching. A path is an \emph{$M$-alternating path}\index{M@$M$-alternating path}
    if the edges along the path alternate between $M$ and $\overline{M} = E \setminus M$.
\end{definition}

\begin{definition}
    An $M$-alternating path is called an \emph{$M$-augmenting path}\index{M@$M$-augmenting path}
    if both ends of the path are uncovered by $M$.
\end{definition}

\datum{2024-10-03}

\begin{proposition}
    Let $G$ be a graph and $M$ a matching. If there exists an $M$-augmenting path 
    $P$, then $M$ is not a maximum matching.
\end{proposition}

\begin{proof}
    We can construct a bigger matching $M' = M \triangle E(P)$, where 
    $\triangle$ is the disjunctive union.
\end{proof}

\begin{theorem}[König]
    Let $G$ be a bipartite graph. Then the following holds:
    \begin{itemize}[(a)]
        \item $\alpha'(G) = \beta(G)$.
        \item If $M$ is a matching witn no $M$-augmenting path, then $M$ is a 
        maximum matching.
    \end{itemize}    
\end{theorem}

%\begin{proof}
%    TO DO
%\end{proof}

\begin{remark}
    There also exist graphs with $\alpha'(G) = \beta(G)$ that are not bipartite.
\end{remark}

\begin{corollary}
    If $G$ is a bipartite graph, then $\alpha(G) = \beta'(G)$
\end{corollary}

\begin{proof}
    We have
    \[
        \alpha(G) = n(G) - \beta(G) = n(G) - \alpha'(G) = \beta'(G),
    \]
    where the latter two equalities are due to König's and Gallai's 
    theorem respectively.
\end{proof}

\begin{definition}
    Let $G$ be a bipartite graph with partite classes $A$ and $B$. 
    \emph{Hall's condition}\index{Hall's condition} (HC) holds for $A$,
    if 
    \begin{equation}
        \forall S \subseteq A.~\abs{S} \le \abs{N(S)},
        \tag{HC}
        \label{eq:HC}
    \end{equation}
    where $N(S)$ is the neighbourhood of $S$.
\end{definition}

\begin{theorem}[Hall]\index{Hall's theorem}
    Let $G = (A \cup B, E)$ be a bipartite graph. A matching that covers $A$ 
    exists if and only if Hall's condition holds for $A$.
\end{theorem}

%\begin{proof}
%    TO DO
%\end{proof}

\begin{definition}
    A matching $M$ is a \emph{perfect matching}\index{perfect matching} if 
    it covers all vertices.
\end{definition}

\begin{corollary}
    Let $G$ be a bipartite graph. There exists a perfect matching of $G$ 
    if and only if $\abs{A} = \abs{B}$ and $A$ satisfies Hall's condition.
\end{corollary}

\begin{definition}
    Let $S \subseteq A$. The \emph{deficiency}\index{deficiency} of $S$ is 
    defined as $\defi(S) = \abs{S} - \abs{N(S)}$.
\end{definition}

\begin{theorem}
    Let $G = (A \cup B, E)$ be a bipartite graph and $M$ a matching. Then at 
    most
    \[
        \abs{A} - \max_{S \subseteq A}(\defi(S))
    \]
    vertices of $A$ are covered.
\end{theorem}

%\begin{proof}
%    Left as an exercise.
%\end{proof}

\begin{theorem}
    If $G$ is a regular bipartite graph, then $G$ has a perfect matching.
\end{theorem}

%\begin{proof}
%    TO DO
%\end{proof}

\datum{2024-10-10}

\begin{theorem}
    Let $M$ be a matching in $G$. There exists an $M$-augmenting path in $G$ if and only of $M$ is not maximum.
\end{theorem}

%\begin{proof}
%    TO DO
%\end{proof}

The \href{https://en.wikipedia.org/wiki/Blossom_algorithm}{Blossom algorithm} is a known algorithm in $O(n\sqrt{n})$ that finds $M$-augmenting paths
(in polynomial time). It also provides a maximum matching, and it allows us to find $\alpha'(G)$ and $\beta'(G)$ in polynomial time.

Let $o(G)$ be the number of odd components ($\abs{V(C)} \equiv 1 \pmod{2}$) in $G$.

\begin{theorem}{Tutte}\index{Tutte's theorem}
    A graph $G$ has a perfect matching if and only if 
    \begin{equation}
        \forall S\subseteq V(G).~\abs{S}\ge o(G \setminus S). 
        \tag{TC}
        \label{eq:TC}
    \end{equation}
\end{theorem}

The condition \eqref{eq:TC} is called \emph{Tutte's condition}.\index{Tutte's condition}

%\begin{proof}
%    TO DO
%\end{proof}

\begin{remark}
    In bipartite graphs \eqref{eq:TC} implies \eqref{eq:HC}.
\end{remark}

\begin{theorem}[Berge-Tutte formula]\index{Berge-Tutte formula}
    A maximum matching in a graph $G$ leaves exactly
    \[
    \max_{S \subseteq V(G)} \left(o(G \setminus S) - \abs{S}\right).
    \]
    vertices uncovered. Equivalently,
    \[
    \alpha'(G) = \frac{1}{2}\left(n - \max_{S \subseteq V(G)} \set{o(G \setminus S) - |S|}\right).
    \]
\end{theorem}

%\begin{proof}
%    Left as an exercise.
%\end{proof}