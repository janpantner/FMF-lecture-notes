\documentclass[12pt, a4paper]{article}

\usepackage[black]{../../general}

\begin{document}
\Huge
\begin{center}
    Algebra 3 \\    
\end{center}
\normalsize
\section*{Vaje 1}

\textbf{Naloga 1.}
Dokaži, da je število $\sqrt{2} + i\sqrt{3}$ algebraično. Poišči njegov minimalni polinom.

\textbf{Naloga 2.}
Določi $[\mathbb{Q}(\sqrt{2} + \sqrt[3]{2}) : \mathbb{Q}]$, 
$[\mathbb{Q}(\sqrt{2} + \sqrt[4]{2}) : \mathbb{Q}]$ in 
$[\mathbb{Q}(\sqrt[6]{2}) : \mathbb{Q}(\sqrt{2})]$.

Če je polje $K$ razširitev polja $F$ (tj.~$F \subseteq K$), včasih pišemo $K/F$. 
Tako lahko npr.~namesto ``$K$ je algebraična razširitev $F$'' 
pišemo ``$K/F$ je algebraična razširitev".

\textbf{Naloga 3.}
Naj bo $K/\mathbb{Q}$ kvadratična razširitev (tj.~razširitev stopnje 2). 
Dokaži, da obstaja enolično določeno celo število $a \in \mathbb{Z}$, 
$a \neq 1$, brez kvadratov, za katerega je $K \cong \mathbb{Q}(\sqrt{a})$.

\textbf{Naloga 4.}
Naj bo $p \in \mathbb{N}$ praštevilo in $\zeta = e^{2\pi i / p}$ 
primitivni $p$-ti koren enote. Dokaži, da je $\zeta$ algebraično število
in določi stopnjo $[\mathbb{Q}(\zeta) : \mathbb{Q}]$.

\end{document}