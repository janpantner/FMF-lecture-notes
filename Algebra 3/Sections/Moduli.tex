\section{Moduli}
\subsection{Vložitev kolobarja v kolobar endomorfizmov}
\datum{2024-11-06}

Naj bo $M$ aditivna grupa. Množica endomorfizmov $\End(M)$ skupaj z operacijama
\begin{align*}
    (\varphi + \psi)(v) &= \varphi(v) + \psi(v) \quad \text{in} \\
    (\varphi \cdot \psi)(v) &= \varphi(\psi(v))
\end{align*}
je kolobar.

\begin{izrek}
    Vsak kolobar lahko vložimo v kolobar endomorfizmov neke aditivne grupe.
    \label{thm:kolobar}
\end{izrek}

\begin{proof}
    Naj bo $K$ kolobar in $\End(K)$ kolobar endomorfizmov aditivne grupe $(K, +)$. definiramo
    \begin{align*}
        \varphi &\colon K \to \End(K), \\
        a &\mapsto l_a,
    \end{align*}
    kjer je $l_a$ levo množenje: $l_a(x) = ax$. Velja
    \begin{align*}
        \varphi(a+b) &= l_{a+b} = l_a + l_b = \varphi(a) + \varphi(b), \\
        \varphi(a\cdot b) &= l_{a \cdot b} = l_a \circ l_b = \varphi(a) \cdot \varphi(b), \\
        \varphi(1) &= l_1 = \id_K.
    \end{align*}
    Velja še
    \[
        \varphi(a) = 0 \Rightarrow l_a = 0 \Rightarrow l_a(1) = 0 \Rightarrow a = 0,
    \]
    torej je jedro trivialno in res imamo vložitev.
\end{proof}

\begin{izrek}
    Vsako algebro lahko vložimo v algebro endomorfizmov $\End_F(V)$ za neki 
    vektorski prostor $V$.
\end{izrek}

\begin{proof}
    Dokaz je podoben dokazu izreka \ref{thm:kolobar}.
\end{proof}

\begin{posledica}
    Vsako končnorazsežno algebro lahko vložimo v $\End_F(V) \cong M_n(F)$, kjer 
    je $V$ $n$-dimenzionalni vektorski prostor nad $F$.
\end{posledica}

\begin{primer}
    Naj bo $A$ $n$-razsežna realna algebra. Ali obstajata takšna $s, t \in A$, da velja 
    $st - ts = 1$?

    Po posledici je to ekvivalentno obstoju $S, T \in M_n(\R)$, kjer velja $ST - TS = I$.
    To ni mogoče, saj velja
    \[
        0 = \tr(ST - TS) \neq \tr(I) = n.
    \]
\end{primer}

\subsection{Definicija modula}

\begin{definicija}
    Naj bo $K$ kolobar. Množica $M$ skupaj z binarno operacijo seštevanja $+$ in zunanjo
    binarno operacijo $K \times M \to M$, $(a, u) \mapsto au$ imenovano 
    \emph{modulsko množenje}\index{modulsko množenje} (tudi skalarno množenje), 
    se imenuje \emph{(levi) modul}\index{modul}\index{levi modul} nad $K$ ali \emph{$K$-modul}, če velja:
    \begin{itemize}
        \item $(M, +)$ je Abelova grupa,
        \item $\forall a \in K.~\forall u, v \in M.~a(u+v)=au + av$,
        \item $\forall a, b \in K.~\forall u \in M.~(a+b)u = au + bu$,
        \item $\forall a, b \in K.~\forall u \in M.~(ab)u = a(bu)$,
        \item $\forall u \in M.~1u = u$.
    \end{itemize}   
\end{definicija}

\begin{opomba}
    Analogno lahko definiramo tudi \emph{desni modul}.\index{desni modul}
\end{opomba}

\begin{opomba}
    Če je $M$ $K$-modul, je $\varphi \colon K \to \End(M)$, $\varphi(a)(u) = au$, 
    homomorfizem kolobarjev.

    Obratno, če je $\varphi \colon K \to \End(M)$ homomorfizem kolobarjev, postane 
    $M$ $K$-modul, če vpeljemo $au := \varphi(a)(u)$.
\end{opomba}

\begin{primer}
    \begin{enumerate}[(1)]
        \item Vektorski prostor nad poljem $F$ je $F$-modul.
        \item Vsaka Abelova (aditivna) grupa je $\Z$-modul. Obratno, $\Z$-modul je 
        aditivna grupa.
        \item Vsak kolobar $K$ je $K$-modul, če za modulsko množenje vzamemo običajno 
        množenje v kolobarju.
        \item Če je $I$ levi ideal $K$, ga lahko obravnavamo kot levi $K$-modul.
        \item Če je $K$ podkolobar $K'$, je $K'$ $K$-modul.
        \item Naj bo $K = M_n(F)$ in $M = F^n$. Potem je $M$ $K$-modul za običajno 
        množenje matrike s stolpcem.
    \end{enumerate}
\end{primer}

\subsection{Osnovni pojmi teorije modulov}
\subsubsection*{Podmoduli}

\begin{definicija}
    Podmnožica $N$ $K$-modula $M$ je \emph{podmodul},\index{podmodul} če je za isti operaciji tudi sama $K$-modul.
\end{definicija}

Ekvivalentno
\begin{align*}
    &\forall a,b \in K.~\forall u,v \in N.~au + bv \in N
    \intertext{oziroma}
    &(\forall u,v \in N.~u+v \in N) \land (\forall a \in K. \forall t \in N.~at \in N).
\end{align*}

\begin{primer}
    \begin{enumerate}[(1)]
        \item Če je $K$ polje, so podmoduli podprostori.
        \item Če je $K = \Z$, so podmoduli podgrupe.
        \item Podmoduli $K$-modula $K$ so levi ideali.
        \item Množici $\set{0}$ in $M$ sta vedno podmodula modula $M$.
    \end{enumerate}
\end{primer}

\begin{trditev}
    Če sta $N_1$ in $N_2$ podmodula, sta podmodula tudi
    \[
        N_1 + N_2 = \setb{v_1 + v_2}{v_i \in N_i}
    \]
    in $N_1 \cap N_2$.
\end{trditev}

% To je enostavno preveriti.

\begin{definicija}
    Modul $M \neq \set{0}$, ki nima drugih podmodulov poleg $\set{0}$ in $M$, se imenuje 
    \emph{enostavni modul}.\index{enostavni modul}
\end{definicija}

\begin{primer}
    \begin{enumerate}[(1)]
        \item Če je $K$ polje, so enostavni moduli $1$-razsežni prostori.
        \item Če je $K = \Z$, so enostavni moduli $\Z_p$, kjer je $p$ praštevilo.
        \item Naj bo $K = M_n(F)$ in $M = F^n$. Naj bo $N \neq \set{0}$ podmodul $M$ in 
        $x \in N$. Velja
        \[
            \forall y \in M.~\exists A \in K.~Ax = y.
        \]
        Torej ni pravega podmodula -- $M$ je enostaven $K$-modul.
    \end{enumerate}
\end{primer}

\subsubsection*{Homomorfizmi modulov}

%
%

\subsubsection*{Kolkobarji endomorfizmov in Schurova lema}

\begin{lema}[Schur]
    Če je $M$ enostaven $K$-modul, je $\End_K(M)$ obseg.
\end{lema}

\begin{proof}
    Naj bo $\varphi \End_K(M)$. Upoštevamo, da sta $\ker \varphi$ in $\im \varphi$ podmodula 
    enostavnega modula. Torej je $\varphi = 0$ ali pa je $\varphi$ bijektiven endomorfizem.
\end{proof}