\section{Moduli}
\subsection{Vložitev kolobarja v kolobar endomorfizmov}
\datum{2024-11-06}

Naj bo $M$ aditivna grupa. Množica endomorfizmov $\End(M)$ skupaj z operacijama
\begin{align*}
    (\varphi + \psi)(v) &= \varphi(v) + \psi(v) \quad \text{in} \\
    (\varphi \cdot \psi)(v) &= \varphi(\psi(v))
\end{align*}
je kolobar.

\begin{izrek}
    Vsak kolobar lahko vložimo v kolobar endomorfizmov neke aditivne grupe.
    \label{thm:kolobar}
\end{izrek}

\begin{proof}
    Naj bo $K$ kolobar in $\End(K)$ kolobar endomorfizmov aditivne grupe $(K, +)$. definiramo
    \begin{align*}
        \varphi &\colon K \to \End(K), \\
        a &\mapsto l_a,
    \end{align*}
    kjer je $l_a$ levo množenje: $l_a(x) = ax$. Velja
    \begin{align*}
        \varphi(a+b) &= l_{a+b} = l_a + l_b = \varphi(a) + \varphi(b), \\
        \varphi(a\cdot b) &= l_{a \cdot b} = l_a \circ l_b = \varphi(a) \cdot \varphi(b), \\
        \varphi(1) &= l_1 = \id_K.
    \end{align*}
    Velja še
    \[
        \varphi(a) = 0 \Rightarrow l_a = 0 \Rightarrow l_a(1) = 0 \Rightarrow a = 0,
    \]
    torej je jedro trivialno in res imamo vložitev.
\end{proof}

\begin{izrek}
    Vsako algebro lahko vložimo v algebro endomorfizmov $\End_F(V)$ za neki 
    vektorski prostor $V$.
\end{izrek}

\begin{proof}
    Dokaz je podoben dokazu izreka \ref{thm:kolobar}.
\end{proof}

\begin{posledica}
    Vsako končnorazsežno algebro lahko vložimo v $\End_F(V) \cong M_n(F)$, kjer 
    je $V$ $n$-dimenzionalni vektorski prostor nad $F$.
\end{posledica}

\begin{primer}
    Naj bo $A$ $n$-razsežna realna algebra. Ali obstajata takšna $s, t \in A$, da velja 
    $st - ts = 1$?

    Po posledici je to ekvivalentno obstoju $S, T \in M_n(\R)$, kjer velja $ST - TS = I$.
    To ni mogoče, saj velja
    \[
        0 = \tr(ST - TS) \neq \tr(I) = n.
    \]
\end{primer}

\subsection{Definicija modula}

\begin{definicija}
    Naj bo $K$ kolobar. Množica $M$ skupaj z binarno operacijo seštevanja $+$ in zunanjo
    binarno operacijo $K \times M \to M$, $(a, u) \mapsto au$ imenovano 
    \emph{modulsko množenje}\index{modulsko množenje} (tudi skalarno množenje), 
    se imenuje \emph{(levi) modul}\index{modul}\index{levi modul} nad $K$ ali \emph{$K$-modul}, če velja:
    \begin{itemize}
        \item $(M, +)$ je Abelova grupa,
        \item $\forall a \in K.~\forall u, v \in M.~a(u+v)=au + av$,
        \item $\forall a, b \in K.~\forall u \in M.~(a+b)u = au + bu$,
        \item $\forall a, b \in K.~\forall u \in M.~(ab)u = a(bu)$,
        \item $\forall u \in M.~1u = u$.
    \end{itemize}   
\end{definicija}

\begin{opomba}
    Analogno lahko definiramo tudi \emph{desni modul}.\index{desni modul}
\end{opomba}

\begin{opomba}
    Če je $M$ $K$-modul, je $\varphi \colon K \to \End(M)$, $\varphi(a)(u) = au$, 
    homomorfizem kolobarjev.

    Obratno, če je $\varphi \colon K \to \End(M)$ homomorfizem kolobarjev, postane 
    $M$ $K$-modul, če vpeljemo $au := \varphi(a)(u)$.
\end{opomba}

\begin{primer}
    \begin{enumerate}[(1)]
        \item Vektorski prostor nad poljem $F$ je $F$-modul.
        \item Vsaka Abelova (aditivna) grupa je $\Z$-modul. Obratno, $\Z$-modul je 
        aditivna grupa.
        \item Vsak kolobar $K$ je $K$-modul, če za modulsko množenje vzamemo običajno 
        množenje v kolobarju.
        \item Če je $I$ levi ideal $K$, ga lahko obravnavamo kot levi $K$-modul.
        \item Če je $K$ podkolobar $K'$, je $K'$ $K$-modul.
        \item Naj bo $K = M_n(F)$ in $M = F^n$. Potem je $M$ $K$-modul za običajno 
        množenje matrike s stolpcem.
    \end{enumerate}
\end{primer}

\subsection{Osnovni pojmi teorije modulov}
\subsubsection*{Podmoduli}

\begin{definicija}
    Podmnožica $N$ $K$-modula $M$ je \emph{podmodul},\index{podmodul} če je za isti operaciji tudi sama $K$-modul.
\end{definicija}

Ekvivalentno
\begin{align*}
    &\forall a,b \in K.~\forall u,v \in N.~au + bv \in N
    \intertext{oziroma}
    &(\forall u,v \in N.~u+v \in N) \land (\forall a \in K. \forall t \in N.~at \in N).
\end{align*}

\begin{primer}
    \begin{enumerate}[(1)]
        \item Če je $K$ polje, so podmoduli podprostori.
        \item Če je $K = \Z$, so podmoduli podgrupe.
        \item Podmoduli $K$-modula $K$ so levi ideali.
        \item Množici $\set{0}$ in $M$ sta vedno podmodula modula $M$.
    \end{enumerate}
\end{primer}

\begin{trditev}
    Če sta $N_1$ in $N_2$ podmodula, sta podmodula tudi
    \[
        N_1 + N_2 = \setb{v_1 + v_2}{v_i \in N_i}
    \]
    in $N_1 \cap N_2$.
\end{trditev}

% To je enostavno preveriti.

\begin{definicija}
    Modul $M \neq \set{0}$, ki nima drugih podmodulov poleg $\set{0}$ in $M$, se imenuje 
    \emph{enostavni modul}.\index{enostavni modul}
\end{definicija}

\begin{primer}
    \begin{enumerate}[(1)]
        \item Če je $K$ polje, so enostavni moduli $1$-razsežni prostori.
        \item Če je $K = \Z$, so enostavni moduli $\Z_p$, kjer je $p$ praštevilo.
        \item Naj bo $K = M_n(F)$ in $M = F^n$. Naj bo $N \neq \set{0}$ podmodul $M$ in 
        $x \in N$. Velja
        \[
            \forall y \in M.~\exists A \in K.~Ax = y.
        \]
        Torej ni pravega podmodula -- $M$ je enostaven $K$-modul.
    \end{enumerate}
\end{primer}

\subsubsection*{Homomorfizmi modulov}

\begin{definicija}
    Naj bosta $M$ in $M'$ $K$-modula. Preslikava $\varphi \colon M \to M'$ je 
    \emph{homomorfizem modulov},\index{homomorfizem modulov} če velja 
    $\varphi(u+v) = \varphi(u) + \varphi(v)$ in $\varphi(au) = a \varphi(u)$.
    Homomorfizme modulov imenujemo tudi \emph{linearne preslikave}\index{linearna preslikava}
    oziroma $K$-linearne preslikave.\index{K@$K$-linearna preslikava}
\end{definicija}

Ekvivalentno mora veljati $\varphi(au + bv) = a\varphi(u) + b \varphi(v)$.

Seveda velja, da je inverz izomorfizma izomorfizem in da je kompozitum homomorfizmov 
homomorfizem. Na standarden način definiramo \emph{jedro} in \emph{sliko} homomorfizma.
Jedro in slika sta podmodula.

\begin{primer}
    \begin{enumerate}
        \item Če je $K$ polje, so homomorfizmi ``običajne'' linearne preslikave.
        \item Če je $K = \Z$, so homomorfizmi aditivne preslikave -- homomorfizmi
        aditivnih grup.
        \item Naj bo $I$ levi ideal $K$. Naj bo $c \in I$. Preslikava $\varphi \colon I \to I$, 
        $u \mapsto uc$, je homomorfizem.
    \end{enumerate}
\end{primer}

\subsubsection*{Kolobarji endomorfizmov in Schurova lema}

Naj bo $M$ $K$-modul. Potem je množica vseh endomorfizmov $M$, $\End_K(M)$, kolobar
za običajno seštevanje in komponiranje kot množenje.

Velja, da je $\varphi \in \End_K(M)$ bijektivna preslikavava natanko tedaj, kadar je avtomorfizem oziroma 
natanko tedaj, kadar je $\varphi$ obrnljiv element $\End_K(M)$.

\begin{lema}[Schur]
    Če je $M$ enostaven $K$-modul, je $\End_K(M)$ obseg.
\end{lema}

\begin{proof}
    Naj bo $\varphi \End_K(M)$. Upoštevamo, da sta $\ker \varphi$ in $\im \varphi$ podmodula 
    enostavnega modula. Torej je $\varphi = 0$ ali pa je $\varphi$ bijektiven endomorfizem.
\end{proof}

\subsubsection*{Kvocientni moduli}
\datum{2024-11-13}

\begin{definicija}
    Naj bo $N$ podmodul $K$-modula $M$. Potem 
    \[
        M/N := \setb{u+N}{u \in M}
    \]
    postane $K$-modul, če vpeljemo
    \begin{align*}
        (u+N) + (v+n) = (u+v) + N, 
        a(u + N) = au + N.
    \end{align*}
    Imenujemo ga \emph{kvocientni modul}.\index{kvocientni modul}
\end{definicija}

Preslikava $\Pi \colon M \to M/N$, $\Pi(u) = u + N$ je epimorfizem modulov. Imenujemo ga 
\emph{kanonični epimorfizem}.\index{kanonični epimorfizem}

Tudi za module velja izrek o izomorfizmu. 
\begin{izrek}[o izomorfizmu]\index{izrek!o izomorfizmu}
    Naj bo $\varphi \colon M \to M'$ homomorfizem modulov. Velja
    \[
        M/_{\ker\varphi} \cong \im \varphi.
    \]
\end{izrek}

\begin{primer}
    \begin{enumerate}
        \item Če je $K$ polje, so kvocientni moduli kvocientni prostori.
        \item Če je $K = \Z$, so kvocientni moduli kvocientne grupe.
        \item Podmodul $K$-modula $K$ je levi ideal $I$. Množica 
        \[
            K/I = \setb{a \in I}{a \in K}
        \]
        je aditivna grupa $K/I$ z modulsko operacijo
        \[
            a(b+I) = ab + I.
        \]
    \end{enumerate}
\end{primer}

\subsubsection*{Direktne vsote modulov}


Naj bodo $N_1, \dots, N_s$ $K$-moduli. Potem $N_1 \times \dots \times N_s$ 
postane $K$-modul, če definiramo
\begin{align*}
    (u_1, \dots, u_s) + (v_1, \dots, v_s) &:= (u_1 + v_1, \dots, u_s + v_s), \\
    a(u_1, \dots, u_s) &:= (au_1, \dots, au_s).
\end{align*}
Imenujemo ga \emph{zunanja direktna vsota}\index{zunanja direktna vsota} modulov 
$N_1, \dots, N_s$. Oznaka $N_1 \oplus \dots \oplus N_s$.

\begin{primer} % To številčenje je precej slabo.
    \begin{enumerate}
        \item Če je $K$ polje, je to direktna vsota vektorskih prostorov.
        \item Če je $K = \Z$, je to direktna vsota aditivnih grup. 
    \end{enumerate}
\end{primer}

Naj bodo $N_1, \dots, N_s$ podmoduli $K$-modula $M$. Če velja 
\begin{enumerate}
    \item $M = N_1 + \dots + N_s = \setb{n_1 + \dots + n_s}{n_i \in N_i}$ in
    \item $N_i \cap (N_1 +  \dots + N_{i-1} + N_{i+1} + \dots + N_s) = \set{0}$ za 
    $i \in \set{1, \dots, s}$.
\end{enumerate}
potem je $M$ \emph{notranja direktna vsota} podmulov $N_1, \dots, N_s$.

\begin{trditev}
    
\end{trditev}

\begin{definicija}
    
\end{definicija}

\begin{primer}
    
\end{primer}


\subsubsection*{Generatorji modulov}

\begin{definicija}
    
\end{definicija}

\begin{primer}
    
\end{primer}

\begin{definicija}
    
\end{definicija}

\begin{lema}
    
\end{lema}

\begin{proof}
    
\end{proof}

\begin{definicija}
    
\end{definicija}

%%

\subsection{Baze modulov in prosti moduli}

\begin{definicija}
    Podmnožica $B$ $K$-modula $M$ je \emph{linearno neodvisna},\index{linearna neodvisnost} če 
    za vse različne elemente $e_1, \dots, e_s \in B$ in vse $a_1, \dots, a_s \in K$ velja
    \[
        a_1b_1 + \dots + a_sb_s = 0 \quad \Rightarrow \quad a_1 = \dots = a_s = 0.
    \]
\end{definicija}

\begin{definicija}
    Če je $B$ linearno neodvisna množica in generira modul $M$, ji rečemo \emph{baza}\index{baza}
    modula $M$.
\end{definicija}

Če je $B$ baza, potem za vsak element $u \in M$ obstajajo taki elementi $e_1, \dots, e_s \in B$,
da je $u = a_1e_1 + \dots a_se_s$ za neke (enolilno določene) $a_i \in K$.

Poenostavljeno zapišemo $u = \sum_i a_i e_i$, kjer je $B = \set{e_i}_i$. Tu moramo razumeti, 
da je le končno mnogo $a_i$-jev lahko različnih od $0$.

\begin{primer}
    Končna netrivialna aditivna grupa nima baze, saj nima nepraznih neodvisnih množic. Velja namreč
    \[
        ne = 0 \not\Rightarrow n = 0, 
    \]
    saj ima v končni grupi vsak element končen red.
\end{primer}

\begin{definicija}
    Modul, ki ima bazo, se imenuje \emph{prosti modul}.\index{prosti modul}
\end{definicija}

\begin{primer}
    \begin{enumerate}
        \item Naj bo $K$ kolobar. Potem je $K^s = K \oplus \dots \oplus K$ prost $K$-modul 
        z bazo $\set{(1,0,\dots,0), (0, 1, 0, \dots, 0), (0,\dots,0,1)}$. 

        Če je $M$ prost $K$-modul z bazo $\set{e_1, \dots, e_s}$, je $M \cong K^s$ 
        (izomorfizem: $a_1e_1 + \dots + a_se_s \mapsto (a_1, \dots, a_s)$).

        \item Naj bo $K$ kolobar. Potem je $K[X]$ prost $K$-modul z bazo $\set{1, x, x^2, \dots}$.
    \end{enumerate}
\end{primer}

\begin{definicija}
    Prostemu $\Z$-modulu pravimo \emph{prosta Abelova grupa}.\index{prosta Abelova grupa}
\end{definicija}

\begin{opomba}
    To ni isto kot prosta grupa.
\end{opomba}

\begin{primer}
    Primer proste Abelove grupe je $\Z^s$.
\end{primer}

\begin{opomba}
    Podmodul prostega modula ni nujno prost.
\end{opomba}

\begin{primer}
    Modul $\Z_4$ je prost $\Z_4$-modul. Njegov podmodul $2\Z_4 = \set{0, 2}$ ni prost.
\end{primer}

\begin{opomba}
    Če je $M$ prost modul in $N$ podmodul, tedaj $M/N$ ni nujno prost.
\end{opomba}

\begin{primer}
    Modul $\Z$ je prost $\Z$-modul in $n\Z$ je prost podmodul. Modul $M/N = \Z_n$ pa ni prost 
    $\Z$-modul.
\end{primer}

\begin{opomba}
    Če ima prost modul bazo z $n$ elementi, ni nujno res, da je vsaka linearno neodvisna 
    množica z $n$ elementi tudi baza.
\end{opomba}

\begin{primer}
    Modul $\Z$ ima bazo $\set{-1}$, množica $\set{2}$ pa ni baza.
\end{primer}

\begin{opomba}
    Obstajajo kolobarji $K$ (nujno nekomutativni), za katere velja $K^s \cong K^t$ tudi, če
    $s \neq t$.
\end{opomba}

%\datum{2024-11-20}