\appendix
\section{Naloge}
\subsection*{Vaje 1}
\datum{2024-10-7}

\begin{enumerate}
    \item Dokaži, da je število $\sqrt{2} + i\sqrt{3}$ algebraično. Poišči njegov minimalni polinom.

    \item Določi $[\Q(\sqrt{2} + \sqrt[3]{2}) : \Q]$, $[\Q(\sqrt{2} + 
    \sqrt[4]{2}) : \Q]$ in $[\Q(\sqrt[6]{2}) : \Q(\sqrt{2})]$.

    \item Naj bo $K/\Q$ kvadratična razširitev (tj.~razširitev stopnje $2$). 
    Dokaži, da obstaja enolično določeno celo število $a \in \Z, a \neq 1$, 
    brez kvadratov, za katerega je $K \cong \Q(\sqrt{a})$.

    \item Naj bo $p \in \mathbb{N}$ praštevilo in $\zeta = e^{2\pi i/p}$ primitivni $p$-ti koren enote. 
    Dokaži, da je $\zeta$ algebraično število, in določi stopnjo $[\Q(\zeta) : \Q]$.
\end{enumerate}

\subsection*{Vaje 2}
\datum{2024-10-14}

\begin{enumerate}
    \item Naj bosta $a$ in $b$ algebraična elementa nad poljem $F$. 
    Denimo, da sta stopnji $[F(a) : F]$ in $[F(b) : F]$ tuji si števili. Dokaži, da je
    \[
    [F(a, b) : F] = [F(a) : F][F(b) : F].
    \]
    \item Določi razpadno polje $K$ polinoma $x^5 - 2$ in izračunaj $[K : \Q]$.
    \item Poišči primitiven element za razširitev $\Q(\sqrt{2}, \sqrt{3})/\Q$.
    \item Izračunaj $[\Q(\sqrt{2} + \sqrt{3} + \sqrt{5}) : \Q]$.
    \item Naj bo $\omega$ transcendenten element nad $\Z_2$. 
    Dokaži, da je polinom $f(x) = x^2 - \omega$ nerazcepen nad $\Z_2(\omega)$, a ima dvakratno ničlo.
    \item Naj bo $p$ neko praštevilo. Dokaži, da razširitev $\Z_p(X, Y)/\Z_p(X^p, Y^p)$ ni enostavna.
\end{enumerate}

\subsection*{Vaje 3}
\datum{2024-10-21}

\begin{enumerate}
    \item Pokaži, da je grupa avtomorfizmov realnih števil $\R$, $\Aut(\R)$, trivialna.
    \item Dokaži, da sta edina zvezna avtomorfizma kompleksnih števil $\C$ identiteta in konjugiranje.
    \item Naj bo $[K : F] = 2$. Dokaži, da je $K$ Galoisova razširitev $F$. 
    Določi tudi grupo avtomorfizmov polja $K$, ki fiksirajo vse elemente iz $F$.
    \item Ali je $\Q(\sqrt[4]{2})/\Q$ Galoisova razširitev? 
    Poišči grupo $\Aut(\Q(\sqrt[4]{2})/\Q)$.
    \item Če sta $K/F$ in $L/K$ Galoisovi razširitvi, ali je nujno tudi $L/F$ Galoisova razširitev?
    \item Dokaži, da lahko Galoisovo grupo polinoma stopnje $n$ vložimo v $S_n$ in 
    zato red te Galoisove grupe deli $n!$.
\end{enumerate}


%\newpage
%\subsection*{Vaje 4}
%\datum{2024-10-28}
%
%\begin{enumerate}
%    \item 
%\end{enumerate}

%\newpage
%\subsection*{Vaje 5}
%\datum{2024-11-4}
%
%\begin{enumerate}
%    \item 
%\end{enumerate}