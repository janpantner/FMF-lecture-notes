\section{Vaje}
\subsection*{Vaje 1}
\datum{2024-10-7}

\begin{enumerate}
    \item Dokaži, da je število $\sqrt{2} + i\sqrt{3}$ algebraično. Poišči njegov minimalni polinom.

    \item Določi $[\mathbb{Q}(\sqrt{2} + \sqrt[3]{2}) \colon \mathbb{Q}]$, $[\mathbb{Q}(\sqrt{2} + 
    \sqrt[4]{2}) \colon \mathbb{Q}]$ in $[\mathbb{Q}(\sqrt[6]{2}) \colon \mathbb{Q}(\sqrt{2})]$.

    \item Naj bo $K/\mathbb{Q}$ kvadratična razširitev (tj.~razširitev stopnje $2$). 
    Dokaži, da obstaja enolično določeno celo število $a \in \mathbb{Z}, a \neq 1$, 
    brez kvadratov, za katerega je $K \cong \mathbb{Q}(\sqrt{a})$.

    \item Naj bo $p \in \mathbb{N}$ praštevilo in $\zeta = e^{2\pi i/p}$ primitivni $p$-ti koren enote. 
    Dokaži, da je $\zeta$ algebraično število, in določi stopnjo $[\mathbb{Q}(\zeta) \colon \mathbb{Q}]$.
\end{enumerate}

\textit{Opomba:} Če je polje $K$ razširitev polja $F$ (tj.~$F \subseteq K$), 
včasih pišemo $K/F$. Tako lahko npr.~namesto ``K je algebraična razširitev $F$'' pišemo 
``K/F je algebraična razširitev''.

\newpage