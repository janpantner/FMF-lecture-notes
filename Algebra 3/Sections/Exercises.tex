\appendix
\section{Naloge}
\subsection*{Vaje 1}
\datum{2024-10-7}

\begin{enumerate}
    \item Dokaži, da je število $\sqrt{2} + i\sqrt{3}$ algebraično. Poišči njegov minimalni polinom.

    \item Določi $[\Q(\sqrt{2} + \sqrt[3]{2}) : \Q]$, $[\Q(\sqrt{2} + 
    \sqrt[4]{2}) : \Q]$ in $[\Q(\sqrt[6]{2}) : \Q(\sqrt{2})]$.

    \item Naj bo $K/\Q$ kvadratična razširitev (tj.~razširitev stopnje $2$). 
    Dokaži, da obstaja enolično določeno celo število $a \in \Z, a \neq 1$, 
    brez kvadratov, za katerega je $K \cong \Q(\sqrt{a})$.

    \item Naj bo $p \in \mathbb{N}$ praštevilo in $\zeta = e^{2\pi i/p}$ primitivni $p$-ti koren enote. 
    Dokaži, da je $\zeta$ algebraično število, in določi stopnjo $[\Q(\zeta) : \Q]$.
\end{enumerate}

\subsection*{Vaje 2}
\datum{2024-10-14}

\begin{enumerate}
    \item Naj bosta $a$ in $b$ algebraična elementa nad poljem $F$. 
    Denimo, da sta stopnji $[F(a) : F]$ in $[F(b) : F]$ tuji si števili. Dokaži, da je
    \[
    [F(a, b) : F] = [F(a) : F][F(b) : F].
    \]
    \item Določi razpadno polje $K$ polinoma $x^5 - 2$ in izračunaj $[K : \Q]$.
    \item Poišči primitiven element za razširitev $\Q(\sqrt{2}, \sqrt{3})/\Q$.
    \item Izračunaj $[\Q(\sqrt{2} + \sqrt{3} + \sqrt{5}) : \Q]$.
    \item Naj bo $\omega$ transcendenten element nad $\Z_2$. 
    Dokaži, da je polinom $f(x) = x^2 - \omega$ nerazcepen nad $\Z_2(\omega)$, a ima dvakratno ničlo.
    \item Naj bo $p$ neko praštevilo. Dokaži, da razširitev $\Z_p(X, Y)/\Z_p(X^p, Y^p)$ ni enostavna.
\end{enumerate}

\subsection*{Vaje 3}
\datum{2024-10-21}

\begin{enumerate}
    \item Pokaži, da je grupa avtomorfizmov realnih števil $\R$, $\Aut(\R)$, trivialna.
    \item Dokaži, da sta edina zvezna avtomorfizma kompleksnih števil $\C$ identiteta in konjugiranje.
    \item Naj bo $[K : F] = 2$. Dokaži, da je $K$ Galoisova razširitev $F$. 
    Določi tudi grupo avtomorfizmov polja $K$, ki fiksirajo vse elemente iz $F$.
    \item Ali je $\Q(\sqrt[4]{2})/\Q$ Galoisova razširitev? 
    Poišči grupo $\Aut(\Q(\sqrt[4]{2})/\Q)$.
    \item Če sta $K/F$ in $L/K$ Galoisovi razširitvi, ali je nujno tudi $L/F$ Galoisova razširitev?
    \item Dokaži, da lahko Galoisovo grupo polinoma stopnje $n$ vložimo v $S_n$ in 
    zato red te Galoisove grupe deli $n!$.
\end{enumerate}

\subsection*{Vaje 4}
\datum{2024-10-28}

\begin{enumerate}
    \item Razširitev $K/F$ imenujemo \emph{bikvadratična},\index{bikvadratična razširitev} če je $K = F(\sqrt{a}, \sqrt{b})$ 
    za neka $a, b \in F$ in je $[K:F] = 4$. Poišči Galoisovo grupo bikvadratične razširitve 
    $K/F$ in določi vsa polja $L$, ki ležijo med $F$ in $K$.

    \item Določi vsa podpolja polja $\Q(e^{2\pi i/7})$.
    
    \item Določi vsa podpolja polja $\Q(\sqrt[4]{2})$.
\end{enumerate}

\subsection*{Vaje 5}
\datum{2024-11-4}

\begin{enumerate}
    \item Naj bo $K$ razpadno polje polinoma $x^5 - 2$ nad $\Q$. Določi vse $a \in \Z$, za katere je $\sqrt{a} \in K$.
    \item Naj bo $K/F$ Galoisova razširitev z $[K:F] = 14$. Dokaži, da so so vsa vmesna polja $L$, za katere je 
    $[L:F] = 7$, med seboj izomorfna. Določi tudi, koliko takih vmesnih polj obstaja.
    \item Naj bo $K/F$ Galoisova razširitev. Denimo, da je $\Gal(K/F)$ komutativna grupa. Pokaži, da je 
    vmesno polje $L$ Galoisova razširitev.
    \item Grupi $G$, v kateri je vsaka podgrupa tudi edinka, rečemo \emph{Dedekindova grupa}\index{Dedekindova grupa}.
    Taka grupa $G$ je bodisi komutativna bodisi obstaja epimorfizem $\pi \colon G \to Q_8$, kjer je $Q_8$ 
    kvaternionska grupa. Premisli, kako lahko iz strukture vmesnih polj neke Galoisove razširitve $K/F$ vidimo, 
    da je $\Gal(K/F)$ komutativna grupa.
\end{enumerate}

\subsection*{Vaje 6}
\datum{2024-11-11}

\begin{enumerate}
    \item Naj bo $\alpha = \sqrt{(2 + \sqrt{2})(3 + \sqrt{3})}$ in $K = \Q(\sqrt{2}, \sqrt{3}, \alpha)$. 
    Dokaži, da je razširitev $K/\Q$ Galoisova stopnje $8$ in da je Galoisova grupa $\Gal(K/\Q)$ 
    izomorfna $Q_8 = \{ \pm 1, \pm i, \pm j, \pm k \}$.
    \item Naj bo $K = C^{\infty}(\R)$ kolobar vseh gladkih funkcij na premici in naj bo 
    $\Gamma = (\R^3)^{\R}$ množica vseh vektorskih polj na $\R$. 
    Dokaži, da je $\Gamma$ desni $K$-modul.
    \item Naj bo $T_n(F)$ kolobar vseh zgornje trikotnih matrik nad poljem $F$. 
    Poišči vse podmodule $T_n(F)$-modula $F^n$.
    \item Pokaži, da je neničeln $K$-modul $M$ enostaven natanko tedaj, ko je $M = K a$ za vsak neničeln $a \in K$.
    \item Naj bo $M$ levi $K$-modul. Premisli, da ima množica $M^* = \Hom_K(M, K)$ 
    vseh $K$-modul homomorfizmov iz $M$ v $K$ naravno strukturo desnega $K$-modula. 
    Dokaži, da je $(\prescript{}{K}{K})^* \cong K_K$.
\end{enumerate}

\newpage
\subsection*{Vaje 7}
\datum{2024-11-18}

\begin{enumerate}
    \item Naj bo $M$ levi $K$-modul. Pokaži, da je preslikava $\text{Ev}: M \to M^{**}$,
    podana s predpisom $\text{Ev}(m) := (m \mapsto \varphi(m))$, homomorfizem desnih $K$-modulov.
    Dokaži, da je v primeru, ko je $K$ polje, preslikava $\text{Ev}$ injektivna. Ali je tudi
    surjektivna? Poišči tak kolobar $K$ in modul $M$, da preslikava $\text{Ev}$ ni injektivna.
    \item Dokaži, da je kolobar $K$ izomorfen kolobarju $\prescript{}{K}{K^{**}}$. Sklepaj, da je
    $\prescript{}{K}{K}$ enostaven natanko tedaj, ko je $K$ obseg.
    \item Kateremu znanemu kolobarju je izomorfen $K = \text{End}_{\Z}(\Z \oplus \Z)$? Določi
    vse podmodule $\prescript{}{K}{\Z \oplus \Z}$.
\end{enumerate}

%\newpage
%\subsection*{Vaje 8}
%\datum{2024-11-25}
%
%\begin{enumerate}
%    \item 
%\end{enumerate}

%\newpage
%\subsection*{Vaje 9}
%\datum{2024-12-2}
%
%\begin{enumerate}
%    \item 
%\end{enumerate}

%\newpage
%\subsection*{Vaje 10}
%\datum{2024-12-9}
%
%\begin{enumerate}
%    \item 
%\end{enumerate}