\section{Galoisova teorija}
\subsection{Fundamentalni izrek Galoisove teorije}
\datum{2024-10-16}

% Manjkajo primeri.

\begin{definicija}
    Naj bo $K$ razširitev polja $F$. Grupo avtomorfizmov $K$, ki fiksirajo $F$ označimo z 
    \[
        \Aut(K/F) := \setb{\sigma \in \Aut(K)}{\forall \lambda \in F.~\sigma(\lambda) = \lambda}.
    \]
\end{definicija}

\begin{definicija}
    Naj bo $H \le \Aut(K/F)$. \emph{Polje fiksnih točk} podgrupe $H$ definiramo kot
    \[
        K^H := \setb{x \in K}{\forall \sigma \in H.~\sigma(x)=x}.
    \]
\end{definicija}

\begin{lema}
    Naj bo polje $K$ razširitev polja $F$ s karakteristiko $0$.
    Če je $\sigma \in \Aut(K/F)$ in $a \in K$ ničla $f(x) \in F[x]$, potem je 
    $\sigma(a)$ ničla $f(x)$. 
\end{lema}

%\begin{proof}
%    
%\end{proof}

\begin{opomba}
    Naj bo $K$ končna razširitev polja $F$ s karakteristiko $0$. Po izreku o 
    primitivnem elementu je $K = F(a)$. Vsak avtomorfizmem je tako enolično 
    določen z delovanjem v $a$. Naj bo $p(x)$ minimalni polinom $a$ nad $F$. 
    Sledi, da vsak avtomorfizem, ki fiksira $F$, le permutira ničle $p(x)$, 
    zato je takšnih avtomorfizmov kvečjemu $\deg(p(x))$. Po lemi (ref) % TO DO 
    pa vemo, da jih je natanko $\det(p(x)) = [K:F]$.
\end{opomba}

\datum{2024-10-23}
\begin{lema}
    Naj bo $a \in K$ in naj bodo $a_1=a, a_2, \dots, a_m$ različni
    elementi množice $\setb{\sigma(a)}{\sigma \in H}$. Potem je 
    \[
        p(x) = (x - a_1) (x - a_2) \cdots (x - a_m)
    \]
    minimalni polinom $a$ nad $K^H$.
\end{lema}

%\begin{proof}
%    
%\end{proof}

\begin{lema}
    Velja $\abs{H} = [K : K^H]$ in $[K : F] = \abs{H} \cdot [K^H : F]$.
\end{lema}

%\begin{proof}
%    
%\end{proof}

\begin{izrek}
    Naj bo $K$ končna razširitev polja $F$ s karakteristiko $0$. Naslednji pogoji 
    so ekvivalentni:
    \begin{enumerate}[(i)]
        \item $\abs{\Aut(K/F)} = [K:F]$.
        \item $K^{\Aut(K/F)} = F$.
        \item Vsak nerazcepen polinom v $F[x]$ z ničlo v $K$, razpade v $K$.
        \item $K$ je razpadno polje nekega nerazcepnega polinoma iz $F[x]$.
        \item $K$ je razpadno polje nekega polinoma iz $F[x]$.
    \end{enumerate}
    \label{thm:1}
\end{izrek}

%\begin{proof}
%    
%\end{proof}

\begin{definicija}
    Končna razširitev $K$ polja $F$ s karakteristiko $0$, se imenuje 
    \emph{Galoisova razširitev}\index{Galoisova razširitev}, 
    če ustreza vsem pogojem izreka \ref{thm:1}.
    Tedaj $\Aut(K/F)$ označujemo z $\Gal(K/F)$.

    Če je $K$ razpadno polje polinoma $f(x) \in F[x]$, potem $K$ imenujemo tudi 
    \emph{Galoisova razširitev polinoma $f(x)$}.\index{Galoisova razširitev polinoma} 
\end{definicija}

\begin{opomba}
    Splošneje te pojme vpeljemo za polja s poljubno karakteristiko. Galoisova
    razširitev je normalna in separabilna razširitev.

    Razširitev je \emph{normalna}\index{normalna razširitev}, če zadošča 
    pogoju (iii) iz izreka \ref{thm:1}.

    Razširitev $K/F$ je \emph{separabilna}\index{separabilna razširitev}, 
    če je vsak nerazcepen polinom iz $F[x]$ \emph{separabilen}\index{separabilen polinom}, 
    tj.~vse njegove ničle so enostavne.
\end{opomba}

\begin{izrek}[Fundamentalni izrek Galoisove teorije]
    Naj bo $K$ Galoisova razširitev polja $F$ s karakteristiko $0$. S $\mathcal{F}$
    označimo množico vseh vmesnih polj med $F$ in $K$, z $\mathcal{G}$ pa množico 
    vseh podgrup grupe $G := \Gal(K/F)$.
    \begin{enumerate}[(a)]
        \item Preslikava
        \begin{align*}
            \alpha \colon \mathcal{B} &\to \mathcal{F}, \qquad \alpha(H) = K^H
        \intertext{je bijektivna z inverzom}
            \beta \colon \mathcal{F} &\to \mathcal{B}, \qquad \beta(L) = \Gal(K/L).
        \end{align*}
        \item Če $H$ pripada $L$ -- torej $H = \Gal(K/L)$ oziroma $L = K^H$ -- 
        potem
        \[
            \abs{H} = [K : L] \quad \text{in} \quad [G : H] = [L : F].
        \]
        \item Če $H$ in $H'$ zaporedoma pripadata $L$ in $L'$, potem $H \subseteq H'$
        natanko tedaj, kadar $L \supseteq L'$.
        \item Če $H$ pripada $L$, potem je $H \triangle G$ natanko tedaj, kadar 
        je $L$ Galoisova razširitev $F$. V tem primeru velja $G/H \cong \Gal(L/F)$.
    \end{enumerate}
\end{izrek}

%\begin{proof}
%    
%\end{proof}